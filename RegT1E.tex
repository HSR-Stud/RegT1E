%%%%%%%%%%%%%%%%%%%%%%%%%
% Dokumentinformationen %
%%%%%%%%%%%%%%%%%%%%%%%%%
\newcommand{\titleinfo}{Regelungstechnik 1 - Formelsammlung}
\newcommand{\authorinfo}{Braun \& Co, J.Rast}
\newcommand{\versioninfo}{$Revision: 1 $ - powered by \LaTeX}

%%%%%%%%%%%%%%%%%%%%%%%%%%%%%%%%%%%%%%%%%%%%%
% Standard projektübergreifender Header für
% - Makros 
% - Farben
% - Mathematische Operatoren
%
% DORT NUR ERGÄNZEN, NICHTS LÖSCHEN
%%%%%%%%%%%%%%%%%%%%%%%%%%%%%%%%%%%%%%%%%%%%%
\include{header/header}
%%%%%%%%%%%%%%%%%%%%%%%%%%%%%%%%%%%%%%%%%%%%%%%%%%%%%%%%%%%%%%%%%%%%%%%%%%%%%%%%%%%%%%%%%%%%%%%%
% Neue Befehle und Definitionen                
%%%%%%%%%%%%%%%%%%%%%%%%%%%%%%%%%%%%%%%%%%%%%%%%%%%%%%%%%%%%%%%%%%%%%%%%%%%%%%%%%%%%%%%%%%%%%%%


\begin{document}

\setlength{\parindent}{0pt}
 \section{Technischer Regelkreis}
 \subsection{Definitionen \formelbuch{20}}
	\begin{tabular}{ p{1.5cm} p{15.5cm}}
        System 		&Eine Anordnung von Gebilden, die miteinander in Beziehung stehen
        			und die sich gegenüber der Umgebung abgrenzen lasen\\
        Prozess:	&Die Gesammtheit der zusammenwirkenden Vorgänge in einem System,
        			durch die Materie, Energie, Informationen umgeformt, transportiert und gespeichert wird\\

        SISO:		&Single Input- Single Output\\
        MiMo:		&Multi Input- Multi Output\\
        Steuerung:	&ohne Rückkopplung\\
        			&	-kann bei {\bf stabiler} Strecke {\bf nicht instabil} werden\\
        Regelung:	&mit Rückkopplung (wirkt als Gegenkopplung)\\
        			&	-immer ein Vergleichsglied zwischen Führungsgrösse (Sollwert) und
        			Regelgrösse(Istwert)\\
        			&	-min. ein Vergleichsglied\\
        			&	-kann auf veränderte Störgrössen reagieren\\
        			&	-kann bei {\bf stabiler} Strecke {\bf instabil} werden\\ 

    \end{tabular}

	\begin{tabular}{|p{2.7cm}|p{5.4cm}|l|}
    	\hline
    	{\bf Begriff deutsch}		&{\bf Begriff englisch}	&{\bf Ergänzende}\\
		\hline
		Regelstrecke
    	Reglekreis			&plant controlled system	&Der aufgabengemäss zu beeinflussende
    	Teil des Systems\\
    	\hline
    	Regler				&controller			&Bestehent aus Vergleichsglied und Regelglied\\
    	\hline
    	Regeleinrichtung	&controlling means&\\
    	\hline
    	Reglesignalkreis	&control circuit&\\
    	\hline
    	Vergleichsglied		&comparing element	&Bildet den Fehler (Differenz)
    											e=r-y\\
    	\hline
    	Regelglied			&controller;
    						controlling element	&Berechnet aus dem Fehler die Stellgrösse u\\
    	\hline
    	Stelleinrichtung
    	Verstärker			&actuator;
					    	power amplifier;
    						servo amplifier		&Funktionseinheit, die den Energie- oder Massenstrom
    											lenkt\\
    	\hline
    	Messeinrichtung
    	Messumformer		&measuring unit;
    						transmitter&\\
    	\hline
    	Regelgrösse y		&controlled variable;
    						desiered value&\\
    	\hline
    	Führungsgrösse r	&reference variable;
    						set value&\\
    	\hline
    	Störgrösse z Last	&disturbance variable;
    						load&\\
    	\hline
    	Stellgrösse u		&manipulatetd (correcting)
    						variable&\\
    	\hline
    	Regeldifferenz e
    	Fehler				&deviation;
    						error variable		&e =r-y\\
    	\hline
    	Rückkopplung		&feedback			&Rückführung\\
    	\hline

	\end{tabular}\\
\includegraphics[height=3.6cm]{./bilder/Grundregelkreis_klein.jpg}


	\subsection{Klassifizierung technischer Systeme}	
		\begin{tabular}{|l | l | l|}
        	\hline
        	einfach					&	schwierig			&	\textcolor{red}{\underline{für Regler}}\\
        	\hline
        	\hline
        	statisch				&	\textcolor{red}{\underline{dynamisch}}	&	\\
        	\hline
        	\textcolor{red}{\underline{linear}}			&	nichtlinear			&	Ausnahmen: 2-, 3-Pt.Regler
        	Simulation\\ 
        	\hline
        	\textcolor{red}{\underline{zeitinvariant}}	&	zeitvariant&\\
        	\hline
        	\textcolor{red}{\underline{Zeitkontinuierlich}}&	zeitdiskret	&\\
        	$\dot{y}=\frac{ku-y}{T}$&	$y_{k+1}=a y_k + b u_k$	&\\
        	\hline
        	\textcolor{red}{\underline{wertkontinuierlich}}&	wertdiskret&\\
        	\hline
        	\textcolor{red}{\underline{kausal}}			&	akausal	&\\
        	\hline
        	\textcolor{red}{\underline{konzentrierte}}	&	verteilte			&	zB. Stromleitung: \\
        	\textcolor{red}{\underline{Parameter}}		&	Parameter			&	ein R und ein C/ sehr viele
        	RC-Glieder in Serie\\
        	\hline
        	\textcolor{red}{\underline{deterministisch}}	&	stochastisch		&	vorhersehbar/zufällig\\
        	\hline
        \end{tabular}	
\newpage

\section{Regelungen}
\subsection{Glieder \formelbuch{70}}

	\begin{tabular}{|l|l|l|l|l|l|}
    	\hline
    	\textbf{Benennung}	&\textbf{Funktion}	&\textbf{UTF}	& \textbf{Sprungantw.}	
    		&\textbf{Verlauf Sprungantw.}	&\textbf{Symbol}\\
    	\hline
    	\hline
    	\textbf{P-Glied} \formelbuch{35}		
    	&y(t)=K u(t)		&K			&K
    	&\begin{minipage}{2.4cm}
         \includegraphics[width=2.4cm]{./bilder/verlaufP.jpg}
         \end{minipage}
    	&\begin{minipage}{2.4cm}
         \includegraphics[width=2.4cm]{./bilder/p-Glied.jpg}
         \end{minipage}\\
    	\hline
    	
    	\begin{minipage}{4cm}
		\vspace{0.2cm}   
    	\textbf{I-Glied} \formelbuch{28}  \\
    	\includegraphics[width=3.5cm]{./bilder/OP-Integrator.png}	   
      \end{minipage}	&
    	\parbox{3cm}{$\dot{y}= K u(t)$\\
    				$y=K\int \limits_0^t u(t) dt$\\ \\
    				$K = - \frac{1}{R \cdot C}$}
    				&$\dfrac{K}{s}$
    	&Kt
    	&\begin{minipage}{2.4cm}
         \includegraphics[width=2.4cm]{./bilder/verlaufI.jpg}
         \end{minipage}
    	&\begin{minipage}{2.4cm}
         \includegraphics[width=2.4cm]{./bilder/I-Glied.jpg}
         \end{minipage}\\
    	\hline
    	
    	\textbf{Totzeit-Glied} \formelbuch{41}	
    	&$y(t)=u(t-T_t)$	&$e^{-T_t s}$	&$K \cdot u(t-T_t)$
    	&\begin{minipage}{2.4cm}
         \includegraphics[width=2.4cm]{./bilder/verlaufTt.jpg}\\
         \end{minipage}
    	&\begin{minipage}{2.4cm}
         \includegraphics[width=2.4cm]{./bilder/Tt-Glied.jpg}
         \end{minipage}\\
		\hline
    	 \begin{minipage}{4cm}
			\vspace{0.2cm}
			\textbf{PT$_1$-Glied} \formelbuch{37}\\
    	   	\includegraphics[width=4cm]{./bilder/PT1.png}        
         \end{minipage}
		 &\parbox{3cm}{$T\dot{y}+y=K u(t)$\\\\
		 $T = \dfrac{1}{K_i \cdot K_p}$\\
		 $K = \dfrac{1}{K_p}$}	 
		 &$\dfrac{K}{1+Ts}$
    	 &$K(1-e^{-\frac{t}{T}})$
    	 &\begin{minipage}{2.4cm}
         \includegraphics[width=2.3cm]{./bilder/verlaufPT1.jpg}
         \end{minipage}
    	&\begin{minipage}{2.4cm}
         \includegraphics[width=2.3cm]{./bilder/Pt1-Glied.jpg}
         \end{minipage}\\
    	\hline
    \end{tabular}

	\subsection{Stetigähnliche Regler}
		\subsubsection{Rückkopplung des Reglers \formelbuch{57}}
		\begin{minipage}{9cm}
		\includegraphics[width=8.5cm]{./bilder/ZweipunktreglerMitRueckfuehrung.jpg}
        \end{minipage}
		\begin{minipage}{7.5cm}
        \includegraphics[width=7cm]{./bilder/ZweipunktreglerMitRueckfuehrung_dia.jpg}
        \end{minipage}\\
		Die Rückkopplung mit einem $PT_1$-Glied über dem Zweipunkteregler bewirkt
		einen bedeutend kleineren Rippel. Leider ist der Mittelwert noch unterhalb des
		Sollwertes. Dies kann mit Hilfe eines $PT_1$-Gliedes am Schluss vor der
		Hauptrückkopplung behoben werden.
	
	\subsubsection{Dreipunktregler \formelbuch{59}}
		\begin{minipage}{9cm}
		\includegraphics[width=9cm]{./bilder/Dreipunktregler.jpg}
        \end{minipage}
		\begin{minipage}{7.5cm}
        \includegraphics[width=7.5cm]{./bilder/Dreipunktregler_dia.jpg}
        \end{minipage}\\
		Damit d = 0 wird muss c=$T_u K_I$ sein.
\newpage
		
		
	\subsubsection{Zweipunktregler \formelbuch{46,71}}
		\begin{minipage}{3cm}
 		\includegraphics[height=3cm]{./bilder/Zweipunktregler.jpg}
        \end{minipage}
		\begin{minipage}{15cm}
        Ein Zweipunktregler ist ein unstetig arbeitender Regler mit zwei
        Ausgangszuständen. Je nachdem, ob der Istwert über oder unter dem
        Sollwert liegt, wird der erste oder der zweite Ausgangszustand
        eingenommen.   
        \end{minipage}

		\textbf{Beispiele von Zweipunktereglerschaltungen: \formelbuch{52,53}}\\
		\begin{minipage}{9cm}
			\vspace{.5cm}        
	 		\includegraphics[width=9cm]{./bilder/Zweipunktregler-b+b2.jpg}\\
			Die Anstiegszeit beträgt nach dem Einschwingvorgang:\\
			$t_{ein}=\frac{2a}{(b K_p - q_a)K_i K_m}$ \\ \\
			Die Abfallszeit beträgt nach dem Einschwingvorgang:\\
			$t_{aus}=\frac{2a}{(b K_p + q_a)K_i K_m}$
        \end{minipage}
		\begin{minipage}{9cm}
			\vspace{.5cm}        
			\includegraphics[width=9cm]{./bilder/Zweipunktregler-b+b_dia.jpg}
        \end{minipage}\\
	\vspace{.5cm}
	\hrule
	\vspace{.5cm}
		\begin{minipage}{9cm}
 		\includegraphics[width=9cm]{./bilder/ZweipunktreglerTotglied2.jpg}\\
			Die Anstiegszeit beträgt nach dem Einschwingvorgang:
			$t_{ein}$=$T_g\ln(\frac{e^{-\frac{T_u}{T_g}}(A-a)-b K_s}{A+a-b K_s})+T_u$\\ \\
			Die Abfallszeit beträgt nach dem Einschwingvorgang:
			$t_{aus}$=$T_g\ln(\frac{A+a-b
			K_s+\frac{b K_s}{e^{-\frac{T_u}{T_g}}}}{A-a})$\\
        \end{minipage}
		\begin{minipage}{9cm}
		\includegraphics[width=9cm]{./bilder/ZweipunktreglerTotglied_dia.jpg}			
        \end{minipage}

	\subsection{Stationäre Signale}
% 	Im stationären Fall werden die Fehlersignale jeweils als Null angenommen.
% 	Beachtet werden muss, dass die Integratoren am Ausgang einen beliebigen Wert
% 	annehmen und Totzeitglieder kurzgeschlossen werden können.	
	Für den stationären Fall gelten folgende Bedingungen:
	\begin{itemize}
    	\item \textbf{Fehlersignale} werden jeweils als \textbf{Null} angenommen.
    	\item \textbf{Integratoren} haben am \textbf{Eingang Null}.
    	\item \textbf{Integratoren} haben am \textbf{Ausgang} einen \textbf{beliebigen Wert}.
    	\item \textbf{Totzeitglieder} werden kurzgeschlossen.
    	\item \textbf{PT$_1$-Glieder} wirken als \textbf{P-Glieder} mit $K_P = K_{PT_1}$ behandelt.
  	\end{itemize}

\newpage

%%%%%%%%%%%%%%%%%%%%%%%%%%%%%%%%%%%%%%%%%%%%%%%%%%%%%%%%%%%%%%%%%%%%%%%%%%%%%%%%%%%%%%%%%%%%%%%%%%%%%%%%%%%%%%%%%%%%%%%%%%%%%		
\section{Linearisierung}
	\subsection{LTI-Systeme}
	  	\renewcommand{\arraystretch}{1.5}
		\begin{tabular}{|l|l|}
	    	\hline
	    	\textbf{Linearität} \formelbuch{76} & \textbf{Zeitinvarianz} \formelbuch{82}\\
	    	\hline
	    	$\Phi(x1+x2)=\Phi(x1)+\Phi(x2)$ & $\Phi(x(t-t_0)=\Phi(x)\cdot x(t-t_0)$ \\
	    	$\Phi(c\cdot x)=c\cdot \Phi(x)$ & \\
			\hline    
	    \end{tabular}
	  	\renewcommand{\arraystretch}{1}
	  	
	\subsection{Erwünschte Nichtlinearität \formelbuch{84}}
		In einem Prozess sind Linearitäten meist erwünscht, da sie meist mit
		Gleichungen zu lösen sind.
		Viele Efekte, wie zum Beispiel die Modulation oder so sind aber gerade erst
		durch Nichtlinearitäten möglich. Daher unterscheidet man:\\
	\includegraphics[width=8cm]{./bilder/Liste_Nichtlinearitaeten.jpg}


	\subsection{Erfassen von nicht liniearen Kurven}
	\subsubsection{Messung einer statischen Kennlinie \formelbuch{86}}
	\begin{minipage}{10cm}
		\includegraphics[width=7cm]{./bilder/NichtlinearMitPT1.jpg}   
    \end{minipage}
	\begin{minipage}{7cm}
    	\includegraphics[width=5cm]{./bilder/NichtlinearMitPT1_dia.jpg}
    \end{minipage}\\
		Da das Signal x meist nicht zugänglich ist, muss man den ganzen Prozess
		ausmessen.\\
		Mit einem Dreieck kann man die Kennlinie auslesen. Sie wird jedoch von einer
		``unechten" Hysterese verzert, die eine Hysteresebreite von 2$\alpha$ T, wobei
		$\alpha$ die Rampensteilheit ist und T die Zeitkonstante des PT1-Gliedes.

	\subsubsection{Mathematische Erfassung einer Messkurve \formelbuch{89}}

	\begin{floatingfigure}[r]{12cm}
    \includegraphics[width=10cm,height=3cm]{./bilder/KennlinieMitStuetzwerten.jpg}
	\end{floatingfigure}
   	
     	1. Wahl einer geegneten Funktion:\\
     	Gesucht ist ein Graph, der eine ähnliche Form
		hat wie die Messkurve. Günstig sind Polynome, da sie mathematisch einfach zum
		beschreiben sind, aber auch Sinus- oder Arctan- Funktionen	\\ 
        \\
     	\begin{enumerate}\setcounter{enumi}{1} 
        \item Bestimmung der Parameter:
          \begin{enumerate}
           \item Methode der ausgewählten Punkte:\\
          		\begin{enumerate}
                    \item Gleich viele Messpunkte wie frei zu wählende Parameter der
           			Approximationsfunktionen.
           			\item Einsetzen der Koordinaten der Messpunkte in die Gleichungen.
           		\end{enumerate}
           	\item Methode der kleinsten Fehlerquadrate:
           		\begin{enumerate}
                     \item Mindestens gleich viele Messpunkte wie Parameter der
           		Appxomationsfunktion, können jedoch mehr gewählt werden.
           		\item Fehler bilden zwischen Messpunkt und Funktionspunkt. zB.\\
           		P1:  $e_1=2-y=2-(a(1.5)+b(1.5)^2)$\\
           		P2:  $e_2=5.5-y=5.5-(a(7)+b(7)^2)$\\
           		P3:  $e_3=4.5-y=4.5-(a(4)+b(4)^2)$\\
           		\item Nach Guass ist die beste Näherung, wenn die Summe der
           		Fehlerquadrate minimal wird. \\
           		$\Longrightarrow$ S =$ e_1^2+e_2^2+e_3^2=min$\\
           		$\frac{\delta S}{\delta a}=
           		2 e_1 \frac{\delta e_1}{\delta a}+
           		2 e_2 \frac{\delta e_2}{\delta a}+
           		2 e_3 \frac{\delta e_3}{\delta a}=0$\\
           		$\frac{\delta S}{\delta b}=
           		2 e_1 \frac{\delta e_1}{\delta b}+
           		2 e_2 \frac{\delta e_2}{\delta b}+
           		2 e_3 \frac{\delta e_3}{\delta b}=0$\\
           		\item Nachteil dieser Methode ist, dass es meist einen grossen
           		Rechenaufwand ergibt. Sie ist jedoch genauer als die erste Methode.
           		Ausserdem geht die Approixmationsfunktion nicht durch die Messwerte.
           		\end{enumerate}
		\end{enumerate}
	\end{enumerate}
    
	 
	
	\subsection{Linearisierung}
		\subsubsection{Modelleinschränkung \formelbuch{92}}
			Praktisch alle physikalische Systeme unterliegen gewissen Aussteuergrenzen.
			In diesen können sie linear sein. Es muss darauf achten, dass  das System
			diese nicht überschreitet.
			
		\subsubsection{Arbeitspunkt \formelbuch{93}}
			Wenn eine Kennlinie für einen bestimmten Aussteuerung einen nicht zu grosse
			Krümmung hat, kann man in der Näherung diese durch die Tangente ersetzen, und
			somit wieder linear rechnen. \\ \\
			\textbf{Trigonometrische Funktionen bei Arbeitspunkt $\alpha = 0$}: $\qquad \sin \alpha \approx
			\alpha \qquad \tan \alpha \approx \alpha \qquad \cos \alpha \approx 1$
		\subsubsection{Durch inverse Kennlinie \formelbuch{94}}
		\begin{minipage}{11cm}
			\includegraphics[width=10cm]{./bilder/Kennlinienkompensation.jpg}
        \end{minipage}
		\begin{minipage}{6cm}
        	\includegraphics[width=6cm]{./bilder/Kennlinienkompensation_dia.jpg}
        \end{minipage}\\
			Durch das Vorschalten der Inversen Kennlinie kann man die Nichtlinearitäten
			beheben. Das System als ganzes reagiert nun linear, das Innere des System
			jedoch immer noch nicht, wie man an den Kennlinien gut erkennen kann.
			
		\subsubsection{Durch Gegenkopplung \formelbuch{96}}
				\begin{minipage}{11cm}
			\includegraphics[width=10cm]{./bilder/Gegenkopplung.jpg}
        \end{minipage}
		\begin{minipage}{9cm}
        	\includegraphics[width=6cm]{./bilder/Gegenkopplung_dia.jpg}
        \end{minipage}\\
			Da ein Prozess variierende Koefizienten besizten kann, kann die Kompensation
			nicht immer mit der inversen Kennlinie geschehen. Deshalb ist es oft
			günstiger die Nichtliniearitäten mit einer Gegenkopplung zu kompensieren.
			Diese ist umso wirksamer je grösser die Verstärkung $K_p$ ist.
			Es können jedoch Stabilitätsprobleme auftreten.			

\end{document}
