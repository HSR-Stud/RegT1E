\section{Regelungen}
\subsection{Glieder}

	\begin{tabular}{|l|l|l|l|l|l|}
    	\hline
    	\textbf{Benennung}	&\textbf{Funktion}	&\textbf{UTF}	& \textbf{Sprungantw.}	
    		&\textbf{Verlauf Sprungantw.}	&\textbf{Symbol}\\
    	\hline
    	\hline
    	\textbf{P-Glied} \formelbuch{35} & $y(t)=K u(t)$	& K & K
    	&\begin{minipage}{2.4cm}
         \includegraphics[width=2.4cm,trim=0 0 -5 -5]{./bilder/verlaufP.jpg}
         \end{minipage}
    	&\begin{minipage}{2.4cm}
         \includegraphics[width=2.4cm]{./bilder/p-Glied.jpg}
         \end{minipage}\\
    	\hline
    	\textbf{I-Glied} \formelbuch{28} &
    	\parbox{3cm}{$\dot{y}= K u(t)$\\
    				$y=K\int \limits_0^t u(t) dt$\\}
    				&$\dfrac{K}{s}$
    	&Kt
    	&\begin{minipage}{2.4cm}
         \includegraphics[width=2.4cm,trim=0 0 -5 -5]{./bilder/verlaufI.jpg}
         \end{minipage}
    	&\begin{minipage}{2.4cm}
         \includegraphics[width=2.4cm]{./bilder/I-Glied.jpg}
         \end{minipage}\\
    	\hline
    	
    	\textbf{Totzeit-Glied} \formelbuch{41}	
    	&$y(t)=u(t-T_t)$	&$e^{-T_t s}$	&$K \cdot u(t-T_t)$
    	&\begin{minipage}{2.4cm}
         \includegraphics[width=2.4cm,trim=0 0 -5 -5]{./bilder/verlaufTt.jpg}\\
         \end{minipage}
    	&\begin{minipage}{2.4cm}
         \includegraphics[width=2.4cm]{./bilder/Tt-Glied.jpg}
         \end{minipage}\\
		\hline
    	 \begin{minipage}{4cm}
			\vspace{0.2cm}
			\textbf{PT$_1$-Glied} \formelbuch{37}\\
    	   	\includegraphics[width=4cm]{./bilder/PT1.png}        
         \end{minipage}
		 &\parbox{3cm}{$T\dot{y}+y=K u(t)$\\\\
		 $T = \dfrac{1}{K_i \cdot K_p}$\\
		 $K = \dfrac{1}{K_p}$}	 
		 &$\dfrac{K}{1+Ts}$
    	 &$K(1-e^{-\frac{t}{T}})$
    	 &\begin{minipage}{2.4cm}
         \includegraphics[width=2.3cm]{./bilder/verlaufPT1.jpg}
         \end{minipage}
    	&\begin{minipage}{2.4cm}
         \includegraphics[width=2.3cm]{./bilder/Pt1-Glied.jpg}
         \end{minipage}\\
    	\hline
    \end{tabular} 
 
    \begin{multicols}{3}
    \textbf{P-Glied (nicht inv. Op-Amp)} \\
    \includegraphics[height=2.5cm]{./bilder/OP-Amp.png} \\
    $K = 1 + \frac{R_2}{R_1}$ 
    \columnbreak

	\textbf{P-Glied (inv. Op-Amp)} \\ 
	\includegraphics[height=2.5cm]{./bilder/OP-InvAmp.png} \\
	$K=-\frac{R_2}{R_1}$
    \columnbreak
        
    \textbf{I-Glied} \\ 
    \includegraphics[height=2.5cm]{./bilder/OP-Integrator.png}\\
    $K = - \frac{1}{R \cdot C}$
    
    \end{multicols}
    
    
    \begin{minipage}{11cm}
    \subsection{Stationäre Signale}
	Für den stationären Fall gelten folgende Bedingungen:
	\begin{itemize}
    	\item \textbf{Fehlersignale} werden jeweils als \textbf{Null} angenommen.
    	\item \textbf{Integratoren} haben am \textbf{Eingang Null}.
    	\item \textbf{Integratoren} haben am \textbf{Ausgang} einen \textbf{beliebigen Wert}.
    	\item \textbf{Totzeitglieder} werden kurzgeschlossen.
    	\item \textbf{PT$_1$-Glieder} wirken als \textbf{P-Glieder} mit $K_P = K_{PT_1}$ behandelt.
  	\end{itemize}
  	\end{minipage}
  	\begin{minipage}{7cm}
  		\textbf{Alternative Darstellung des PT$_1$-Glied}\\
  		\includegraphics[width=6cm]{./bilder/PT1_2.png}
  	\end{minipage} \\


	\subsection{Zwei- und Dreipunktregler}
	\subsubsection{Zweipunktregler \formelbuch{46}}
		\begin{minipage}{3cm}
 		\includegraphics[height=3cm]{./bilder/Zweipunktregler.jpg}
        \end{minipage}
		\begin{minipage}{15cm}
        Ein Zweipunktregler ist ein unstetig arbeitender Regler mit zwei
        Ausgangszuständen. Je nachdem, ob der Istwert über oder unter dem
        Sollwert liegt, wird der erste oder der zweite Ausgangszustand
        eingenommen. 
        \end{minipage}
        
		\newpage
		\textbf{Beispiele von Zweipunkte-Reglerschaltungen:}\\ \\
		\textbf{Zweipunktregler mit symmetrischer Kennlinie \formelbuch{52}} \\
		\begin{minipage}{9cm}
			\vspace{.5cm}        
	 		\includegraphics[width=9cm]{./bilder/Zweipunktregler-b+b2.jpg}\\
			Die Anstiegszeit beträgt nach dem Einschwingvorgang:\\
			$t_{ein}=\frac{2a}{(b K_p - q_a)K_i K_m}$ \\ \\
			Die Abfallszeit beträgt nach dem Einschwingvorgang:\\
			$t_{aus}=\frac{2a}{(b K_p + q_a)K_i K_m}$
        \end{minipage}
		\begin{minipage}{9cm}
			\vspace{.5cm}        
			\includegraphics[width=9cm]{./bilder/Zweipunktregler-b+b_dia.jpg}
        \end{minipage}\\

	\vspace{.5cm}
		\textbf{Zweipunktregler an ausgleichender Strecke \formelbuch{53}} \\
		\begin{minipage}{9cm}
 		\includegraphics[width=9cm]{./bilder/ZweipunktreglerTotglied2.jpg}\\
			Die Anstiegszeit beträgt nach dem Einschwingvorgang:
			$t_{ein}$=$T_g\ln(\frac{e^{-\frac{T_u}{T_g}}(A-a)-b K_s}{A+a-b K_s})+T_u$\\ \\
			Die Abfallszeit beträgt nach dem Einschwingvorgang:
			$t_{aus}$=$T_g\ln(\frac{A+a-b
			K_s+\frac{b K_s}{e^{-\frac{T_u}{T_g}}}}{A-a})$\\
        \end{minipage}
		\begin{minipage}{9cm}
		\includegraphics[width=9cm]{./bilder/ZweipunktreglerTotglied_dia.jpg}			
        \end{minipage}
    
    \subsubsection{Rückkopplung des Reglers \formelbuch{57}}
		\begin{minipage}{9cm}
		\includegraphics[width=8.5cm]{./bilder/ZweipunktreglerMitRueckfuehrung.jpg}
        \end{minipage}
		\begin{minipage}{7.5cm}
        \includegraphics[width=7cm]{./bilder/ZweipunktreglerMitRueckfuehrung_dia.jpg}
        \end{minipage}\\
		Die Rückkopplung mit einem $PT_1$-Glied über dem Zweipunkteregler bewirkt
		einen bedeutend kleineren Rippel. Leider ist der Mittelwert noch unterhalb des
		Sollwertes. Dies kann mit Hilfe eines $PT_1$-Gliedes am Schluss vor der
		Hauptrückkopplung behoben werden.
	
	\subsubsection{Dreipunktregler \formelbuch{59}}
		\begin{minipage}{9cm}
		\includegraphics[width=9cm]{./bilder/Dreipunktregler.jpg}
        \end{minipage}
		\begin{minipage}{7.5cm}
        \includegraphics[width=7.5cm]{./bilder/Dreipunktregler_dia.png}
        \end{minipage}\\
		Damit d = 0 wird muss c=$T_u K_I$ sein. Falls ein PT$_1$-Glied vorkommt muss $c=K_I K_s b(T_u + T_G)$ sein.